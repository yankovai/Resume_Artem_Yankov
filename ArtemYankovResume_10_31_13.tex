%%%%%%%%%%%%%%%%%%%%%%%%%%%%%%%%%%%%%%%%%
% Medium Length Graduate Curriculum Vitae
% LaTeX Template
% Version 1.1 (9/12/12)
%
% This template has been downloaded from:
% http://www.LaTeXTemplates.com
%
% Original author:
% Rensselaer Polytechnic Institute (http://www.rpi.edu/dept/arc/training/latex/resumes/)
%
% Important note:
% This template requires the res.cls file to be in the same directory as the
% .tex file. The res.cls file provides the resume style used for structuring the
% document.
%
%%%%%%%%%%%%%%%%%%%%%%%%%%%%%%%%%%%%%%%%%

%----------------------------------------------------------------------------------------
%	PACKAGES AND OTHER DOCUMENT CONFIGURATIONS
%----------------------------------------------------------------------------------------

\documentclass[margin, 10pt]{res} % Use the res.cls style, the font size can be changed to 11pt or 12pt here

\usepackage{helvet} % Default font is the helvetica postscript font
%\usepackage{newcent} % To change the default font to the new century schoolbook postscript font uncomment this line and comment the one above

\setlength{\textwidth}{5.1in} % Text width of the document

\begin{document}

%----------------------------------------------------------------------------------------
%	NAME AND ADDRESS SECTION
%----------------------------------------------------------------------------------------

\moveleft.5\hoffset\centerline{\large\bf Artem Igorevich Yankov} % Your name at the top
 
\moveleft\hoffset\vbox{\hrule width\resumewidth height 1pt}\smallskip % Horizontal line after name; adjust line thickness by changing the '1pt'

\moveleft.5\hoffset\centerline{2200 Fuller Court, Apt. 414B} % Your address
\moveleft.5\hoffset\centerline{Ann Arbor, MI 48105}
\moveleft.5\hoffset\centerline{yankovai@umich.edu}
\moveleft.5\hoffset\centerline{(401) 206-9133}
\moveleft.5\hoffset\centerline{https://sites.google.com/a/umich.edu/yankovai/}
\moveleft.5\hoffset\centerline{GitHub: https://github.com/yankovai}

%----------------------------------------------------------------------------------------

\begin{resume}

%----------------------------------------------------------------------------------------
%	OBJECTIVE SECTION
%----------------------------------------------------------------------------------------
 
\section{OBJECTIVE}  

A position that utilizes computational science and mathematical modeling to solve challenging, real-world problems in a stimulating and fast-paced environment.    

%----------------------------------------------------------------------------------------
%	EDUCATION SECTION
%----------------------------------------------------------------------------------------

\section{EDUCATION}

\textbf{University of Michigan} \hfill Ann Arbor, MI \\
{\sl Ph.D} Nuclear Engineering and Radiological Sciences \hfill Expected 2014

\textbf{Rose-Hulman Institute of Technology} \hfill Terre Haute, IN \\
{\sl B.S.} Mathematics \hfill May, 2010 \\
{\sl B.S.} Physics \\
Minor: Computational Science \\
Clarence P. Sousley Award for demonstration of exceptional performance in the mathematical sciences.
 
%----------------------------------------------------------------------------------------
%	COMPUTER SKILLS SECTION
%----------------------------------------------------------------------------------------

\section{SKILLS} 

{\sl Programming Languages:} Python, R, Fortran, SQL, Bash, \LaTeX, Matlab \\
{\sl Libraries:} Numpy, SciPy, matplotlib, scikit-learn, BeautifulSoup, pandas, ggplot2 \\
{\sl Software:} Maple, Minitab, Dakota, Tableau \\
{\sl Operating Systems:} Unix, Windows, OS X \\
{\sl Machine Learning:} MapReduce, Neural Networks, Logistic/Linear Regression, SVM
 
%----------------------------------------------------------------------------------------
%	PROFESSIONAL EXPERIENCE SECTION
%----------------------------------------------------------------------------------------
 
\section{EXPERIENCE}

{\sl Research Assistant} \hfill July 2010-present \\
University of Michigan, Department of Nuclear Engineering, Ann Arbor, MI

\begin{itemize} \itemsep -2pt % Reduce space between items
\item Developing, analyzing, and applying novel techniques for the uncertainty quantification of computer models for nuclear reactor core simulation.
\item Thesis work in the construction of surrogates for computer models with large numbers of correlated, stochastic inputs.  
\item Coupled software to apply uncertainty quantification techniques to time-dependent reactor simulations in a parallel computing environment.  
\end{itemize}

{\sl Undergraduate Intern} \hfill Summer 2009 \\
Idaho National Laboratory, Idaho Falls, ID 

\begin{itemize} \itemsep -2pt % Reduce space between items
\item Investigated effects of placing gas gap in irradiation capsule experiments at the Advanced Test Reactor. 
\item Used finite element analysis to obtain a uniform specimen temperature profile by adjusting gas gap parameters.
\item Investigated the minimum size of coolant flow channel for design experiments needed to meet thermal-hydraulic safety requirements.
\end{itemize}
 
{\sl Research Experience for Undergraduates} \hfill Summer 2008 \\
Brigham Young University, Department of Mathematics, Provo, UT 
\begin{itemize} 
\item Researched Lagrangian formulations of mechanics with designer conservation laws. 
\end{itemize} 

%----------------------------------------------------------------------------------------
%	PUBLICATIONS
%---------------------------------------------------------------------------------------- 

\section{PUBLICATIONS}

A. Yankov and T. Downar, "Application of Adaptive Hierarchical Sparse Grid Collocation to the Uncertainty Quantification of Nuclear Reactor Simulators," \textit{International Conference on Mathematics and Computational Methods Applied to Nuclear Science and Engineering}, Sun Valley, Idaho, USA, May 5-9, 2013.

A. Yankov, B. Collins, M. Klein, et al., "A Two-Step Approach to Uncertainty Quantification of Core Simulators," \textit{Science and Technology of Nuclear Installations}, vol. 2012, Article ID 767096, 9 pages, 2012. doi:10.1155/2012/767096.

A. Yankov, B. Collins, M. A. Jessee, et al., "A Generalized Adjoint Approach for Quantifying Reflector Assembly Discontinuity Factor Uncertainties," \textit{Proc. PHYSOR 2012}, Knoxville, Tennessee, USA, April 15–20 (2012). %\\
\begin{itemize} 
\item Won best student paper award.
\end{itemize}

A. Yankov, M. Klein, M. A. Jessee, et al., "Comparison of XSUSA and Two-Step Approaches for Full-Core Uncertainty Quantification," \textit{Proc. PHYSOR 2012}, Knoxville, Tennessee, USA, April 15–20 (2012).

%----------------------------------------------------------------------------------------
%	CONFERENCES ATTENDED
%----------------------------------------------------------------------------------------

\section{CONFERENCES \\ ATTENDED}

Reduced Order Modeling in General Relativity \hfill June, 2013 \\
California Institute of Technology, Pasadena, CA

Mathematics and Computation \hfill May, 2013 \\
American Nuclear Society, Sun Valley, ID

Uncertainty Analysis in Best-Estimate Modeling \hfill May, 2012 \\
Karlsruhe Institute of Technology, Karlsruhe, Germany

PHYSOR Advances in Reactor Physics \hfill April, 2012 \\
Oak Ridge National Laboratory, Knoxville, TN

Modeling, Experimentation, and Validation School \hfill July, 2011 \\
Argonne National Laboratory, Argonne, IL

%----------------------------------------------------------------------------------------
%	EXTRA-CURRICULAR ACTIVITIES SECTION
%----------------------------------------------------------------------------------------

\section{EXTRA-CURRICULAR \\ ACTIVITIES} 

Tough Mudder 2012 \\
Detroit Free Press Half-Marathon 2012 \\
Ann Arbor Marathon 2013 (3:45) \\
Detroit Free Press Marathon 2013 (3:35) \\
Ann Arbor Parks and Recreation Ice Hockey \\
Predictive Analytics of Southeast Michigan Meetup Group \\
Blogging \\
Kaggle 

%----------------------------------------------------------------------------------------

\end{resume}
\end{document}