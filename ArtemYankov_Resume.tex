%%%%%%%%%%%%%%%%%%%%%%%%%%%%%%%%%%%%%%%%%
% Medium Length Graduate Curriculum Vitae
% LaTeX Template
% Version 1.1 (9/12/12)
%
% This template has been downloaded from:
% http://www.LaTeXTemplates.com
%
% Original author:
% Rensselaer Polytechnic Institute (http://www.rpi.edu/dept/arc/training/latex/resumes/)
%
% Important note:
% This template requires the res.cls file to be in the same directory as the
% .tex file. The res.cls file provides the resume style used for structuring the
% document.
%
%%%%%%%%%%%%%%%%%%%%%%%%%%%%%%%%%%%%%%%%%

%----------------------------------------------------------------------------------------
%	PACKAGES AND OTHER DOCUMENT CONFIGURATIONS
%----------------------------------------------------------------------------------------

\documentclass[margin, 10pt]{res} % Use the res.cls style, the font size can be changed to 11pt or 12pt here

\usepackage{helvet} % Default font is the helvetica postscript font
%\usepackage{newcent} % To change the default font to the new century schoolbook postscript font uncomment this line and comment the one above

\setlength{\textwidth}{5.1in} % Text width of the document

\begin{document}

%----------------------------------------------------------------------------------------
%	NAME AND ADDRESS SECTION
%----------------------------------------------------------------------------------------

\moveleft.5\hoffset\centerline{\large\bf Artem I. Yankov} % Your name at the top
 
\moveleft\hoffset\vbox{\hrule width\resumewidth height 1pt}\smallskip % Horizontal line after name; adjust line thickness by changing the '1pt'

\moveleft.5\hoffset\centerline{yankovai@umich.edu}
\moveleft.5\hoffset\centerline{\textbf{URL}: https://sites.google.com/a/umich.edu/yankovai/}
\moveleft.5\hoffset\centerline{\textbf{GitHub}: https://github.com/yankovai}
\moveleft.5\hoffset\centerline{(401) 206-9133}

%----------------------------------------------------------------------------------------

\begin{resume}

%----------------------------------------------------------------------------------------
%	TECHNICAL SKILLS
%----------------------------------------------------------------------------------------

\section{TECHNICAL SKILLS} 

{\sl Languages:} Python, R, Fortran, bash, SQLite, Matlab, \LaTeX, Apache Pig \\
{\sl Operating Systems:} OS X, Unix, Windows \\
{\sl Applications and Libraries:} Numpy, SciPy, Pandas, BeautifulSoup, ggplot2, Git, SciKit-Learn, Twitter Streaming API, Tableau, Dakota, Maple  \\
{\sl Other Skills:} Web scraping, linear/logistic regression, support vector machines, uncertainty quantification, numerical linear algebra, reduced order modeling

%----------------------------------------------------------------------------------------
%	PROFESSIONAL EXPERIENCE SECTION
%----------------------------------------------------------------------------------------
 
\section{EXPERIENCE}

{\sl Data Scientist} \hfill July 2010-present \\
Ann Arbor, MI

\begin{itemize} \itemsep -2pt % Reduce space between items
\item StackOverflow Query Tag Extraction
\begin{itemize}
\item Used Python to analyze over 6,000,000 StackOverflow queries and corresponding tags using MapReduce-type framework.  
\item Reformulated raw textual data to a useful form and placed into SQL database for further processing.
\item Developed an algorithm to automatically predict new tags based on similarity to queries in training data set. 
\end{itemize}
\item College Basketball Prediction
\begin{itemize}
\item Scraped a decade's worth of college basketball data from sports-reference.com/cbb using BeautifulSoup and stored results in SQLite database.  
\item Analyzed data and developed a logistic regression model to predict the outcome of unplayed games. 
\item Ranked all college basketball teams based on simulations of predictive model. 
\item Created a visualization of predictive model performance using Tableau.
\end{itemize}
\item Twitter User Cravings
\begin{itemize}
\item Used Twitter Streaming API to investigate cravings of Twitter users. 
\item Utilized Apache Pig to filter and process relevant data. 
\item Created a visualization application using Tableau to present results. 
\end{itemize}
\end{itemize}

{\sl Graduate Student Research Assistant} \hfill July 2010-present \\
University of Michigan, Department of Nuclear Engineering, Ann Arbor, MI

\begin{itemize} \itemsep -2pt % Reduce space between items
\item Worked with Idaho National Laboratory and Sandia National Laboratory to fold high fidelity computer simulations and experimental data towards the creation of optimized nuclear fuel performance models. 
\item Research in how uncertainties in reactor simulation code input parameters propagate to output predictions.
\begin{itemize}
\item Extensive collaboration with researchers from Oak Ridge National Laboratory.
\item Results of research published in leading journal and awarded first prize at a major technical conference.
\end{itemize}
\item Implemented numerical linear algebra routines into primary software used by Nuclear Regulatory Commission to simulate nuclear reactor accident scenarios.
\end{itemize}

%----------------------------------------------------------------------------------------


%----------------------------------------------------------------------------------------
%	EDUCATION SECTION
%----------------------------------------------------------------------------------------

\section{EDUCATION}

\textbf{University of Michigan} \hfill Ann Arbor, MI \\
{\sl Ph.D} Nuclear Engineering and Radiological Sciences \hfill Expected 2014

\textbf{Rose-Hulman Institute of Technology} \hfill Terre Haute, IN \\
{\sl B.S.} Mathematics \hfill May, 2010 \\
{\sl B.S.} Physics \\
Minor: Computational Science \\
Clarence P. Sousley Award for demonstration of exceptional performance in the mathematical sciences.
  
\end{resume}
\end{document}
